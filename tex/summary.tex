\paragraph{}
В ходе выполнения данной работы были реализованы и протестированы различные функции для работы с матрицами. Это включало ввод матрицы, вычисление её следа, поиск элементов по индексам, а также операции сложения, умножения матриц и умножения матрицы на скаляр. Кроме того, были реализованы функции для вычисления определителя матрицы и проверки её обратимости.

Работа над этими задачами позволила нам углубить понимание основ линейной алгебры и её применения в программировании. Особенно интересным оказался метод Гаусса для вычисления определителя матрицы, который\\ демонстрирует применение алгоритмов линейной алгебры в решении сложных задач.

Процесс тестирования, как автоматического, так и ручного, подтвердил корректность реализованных алгоритмов и их способность работать с матрицами различных размеров. Это важно для обеспечения надежности и точности вычислений, что является важным аспектом в машинном обучении и анализе данных.

В целом, выполнение данной работы не только укрепило наши навыки программирования и работы с матрицами, но и заставило задуматься, как полученные знания помогут в будущем. Линейная алгебра является основой для многих алгоритмов машинного обучения, и понимание её принципов и методов является ключом для успешного применения этих алгоритмов на практике.

В будущем мы планируем продолжить изучение машинного обучения, включая более сложные алгоритмы и методы. Уверены, что полученные знания и навыки станут основой для дальнейшего развития в этой области.